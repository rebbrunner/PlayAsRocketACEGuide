\documentclass[../main.tex]{subfiles}

\begin{document}
    The OAM routine is a routine called every animation frame of the gen 1 and gen 2 pokemon games.  It is responsible for updating sprites, allowing them to move and be drawn to screen.  As the function runs every animation frame, if we insert a custom jump point into the routine we can, "hijack," the routine to run ACE every frame.  This is ideal for modifying the game's loop and adding, "event listeners," which will trigger whenever a certain action happens.  In this section we will be going over the OAM routine and how to hijack it.  For a practical example, we will be porting a gen 2 code developed by Timo to add a run button to the generation 1 games.

    \subsection{The OAM Routine}
    TODO

    \subsection{Self installing Hijack}
    TODO

    \subsection{Preventing Crashes}
    TODO

    \subsection{Adding a run Button}
    TODO

    \subsection{Adding a toggle to our script selector}
    TODO
\end{document}

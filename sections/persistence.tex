\documentclass[../main.tex]{subfiles}

\begin{document}
    At this point, every time you reboot your game you will have to re-toggle your hack.  In this section we will be making our OAM hijack persistent - as in the game will automatically restore our hijack after resets.  We will end the section by overriding the encounter music that plays when walking up to a trainer after being spotted.

    \subsection{Map Scripts}
    Every map contains what is called a \mapheader{}.  The \mapheader{} points to all data related to the map, including scripts.  The pointer to scripts is dynamic and stored in RAM as it changes on map load, but at the same time persists through saves because the game remembers what map you were on when you saved.  Additionally, the map script pointer is called once per frame.  All of these facts make this an ideal place to setup a persistent OAM hijack.  Let's run down how this works:

    \begin{enumerate}
        \item When our character saves, we update the map script pointer to point to our self-installing payload.  Otherwise we modify it back.  Our storage location for the original pointer should persist with saves
        \item When our game saves the pointer to our self-installing payload is saved
        \item After we reset, the self-installing payload is loaded and our OAM activates
        \item Since our map script pointer only overwrites when we are saving, the original map script pointer is restored
    \end{enumerate}

    \subsection{Creating a Persistent Hijack}
    Alright let's implement this into our setup.  In your OAM hijack input the following (thanks to TimoVM for providing optimizations):

    \begin{minted}[linenos,breaklines]{text}
F0 B0     ldh a, ($FFB0)
A7        and a, a                          ; If not on overworld, set z flag
21 6E D3  ld hl, wCurMapScriptPointer + 1   ; Target high byte
11 XX XX  ld de, $XXXX                      ; Address where pointer is stored. set to somewhere in unused memory
3A        ldd a, (hl)
20 15     jr nz,15                          ; If we're in a menu, continue
FE ZZ     cp $ZZ                            ; Should point to a WRAM address's high byte
28 1A     jr z                              ; If pointer is already set to custom value, stop here
FA 30 CC  ld a, ($CC30)
FE 73     cp $73
20 13     jr nz                             ; Is SAVE being highlighted in the start menu? if yes, continue
2A        ldi a, (hl)
12        ld (de), a
13        inc de
3A        ldd a, (hl)
12        ld (de), a                        ; Store map script pointer in memory
3E YY     ld a, $YY
22        ldi (hl), a
36 ZZ     ld (hl), $ZZ                      ; Overwrite map script pointer to address ZZYY
FE ZZ     cp $ZZ                            ; If we jumped here, check if pointer has custom value. If yes, restore original map script pointer
20 05     jr nz                             ; Also includes a dirty hack, as long as YY != ZZ, a return is guaranteed if we slide through from the previous section
1A        ld a, (de)
13        inc de
22        ldi (hl), a
1A        ld a, (de)
32        ldd (hl), a
C9        ret
    \end{minted}

    If your setup works, you should be able to run after soft resetting.  Test this is working as expected before continuing.

    \subsection{Modifying Encounter Music}
    To round out this section we're going to modify the encounter music when we enter line of sight for opponents.  At the end of your OAM pay load enter the following:

    \begin{minted}[linenos,breaklines]{text}
18 01		; jr,01		        ; Skip variable
00		    ; x		            ; Variable, should be at DAD2, if not adjust D2 DA throughout this script to the proper location
FA 2D CD	; ld a,(CD2D)	    ; Determine if an encounter has started
D6 6F		; sub a,6F	        ; If encounter is a high number a battle encounter has started
38 18		; jr c,18		    ; Skip if not in an encuonter
FA D2 DA	; ld a,x		    ; Load a variable to keep track of whether music has been started
D6 01		; sub a,01
28 16		; jr z,16		    ; Skip starting music if started
CD 33 22	; call 2233	        ; StopMusic
0E 20		; ld c,20		    ; Bank with music
3E 9C		; ld a,9C		    ; Music address
CD 11 22	; call 2211	        ; PlayMusic
3E 01		; ld a,01
EA D2 DA	; ld x,a		    ; Set variable as music has started
18 05		; jr 05		        ; Skip variable reset
3E 00		; ld a,00
EA D2 DA	; ld x,a		    ; Reset variable
C9		    ; ret
    \end{minted}

    Now when you encounter an opponent trainer you should get the Jessie \& James encounter music.  Alright, we are now entirely done with our setup, now we will move on to some of the more complex codes like editing sprites and custom text for pikachu.
\end{document}
